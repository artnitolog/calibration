\documentclass[12pt]{article}

\usepackage[T2A]{fontenc}
\usepackage[utf8]{inputenc}
\usepackage[english, russian]{babel}
\usepackage[sups]{XCharter}
\usepackage[vvarbb, uprightscript, charter, scaled=1.05]{newtxmath}
\usepackage{enumitem}
\usepackage{verbatim}
\usepackage[justification=centering]{caption}
% \usepackage{caption}
% \captionsetup[figure]{skip=1pt}
\usepackage{microtype}
\usepackage{subcaption}
% \usepackage[style=numeric, sorting=none]{biblatex}
% \addbibresource{refs.bib}
% \usepackage{minted}
% \usepackage{fancyhdr}
% \usepackage{gensymb}
% \usepackage{booktabs}
% \usepackage{ntheorem}
% \usepackage{mathtools}
\usepackage{geometry}
% \usepackage{titling}  
\usepackage{indentfirst}
% \usepackage[normalem]{ulem}
% \useunder{\uline}{\ul}{}
\usepackage{graphicx}
\graphicspath{ {vis/} }

\usepackage[table,xcdraw]{xcolor}
\usepackage{hyperref}
\hypersetup{
    colorlinks=true,
    linkcolor=blue,
    filecolor=magenta,      
    urlcolor=cyan,
}

\geometry{a4paper, textwidth=16cm, textheight=24cm}

\newcommand{\mpl}[2]{
    \begin{figure}[!h]
        \includegraphics[width=0.98\textwidth]{#1}
        \centering
        \caption{#2}
        \label{fig:#1}
     \end{figure}
}

% \title{}

% \author{Васильев Руслан \and{ВМК МГУ, 317 группа}}

\usepackage{tocloft}
\renewcommand{\cftsecleader}{\cftdotfill{\cftdotsep}}

\begin{document}

\begin{titlepage}
\begin{center}
    Московский государственный университет имени М. В. Ломоносова

    \bigskip
    \includegraphics[width=50mm]{msu.eps}

    \bigskip
    Факультет Вычислительной Математики и Кибернетики\\
    Кафедра Математических Методов Прогнозирования\\[10mm]

    % \textsf{\large\textbf{
    %     КУРСОВАЯ РАБОТА СТУДЕНТА 317 ГРУППЫ\\[10mm]
    %     <<Калибровка уверенности>>}
    % }\\[10mm]
    {\large\textbf{
        КУРСОВАЯ РАБОТА СТУДЕНТА 317 ГРУППЫ\\[10mm]
        <<Калибровка уверенности>>}
    }\\[10mm]

    \begin{flushright}
        \parbox{0.5\textwidth}{
            Выполнил:\\
            студент 3 курса 317 группы\\
            \emph{Васильев Руслан Леонидович}\\[5mm]
            Научный руководитель:\\
            д.ф-м.н., профессор\\
            \emph{Дьяконов Александр Геннадьевич}
        }
    \end{flushright}

    % \begin{tabular}{p{0.45\textwidth}p{0.45\textwidth}}
    %     Заведующий кафедрой\newline
    %     Математических Методов\newline
    %     Прогнозирования, академик РАН
    %     &
    %     ~\newline~\newline
    %     \hfill\hbox to 0.45\textwidth{\hrulefill~Ю. И. Журавлёв}
    % \\[20mm]
    %     К защите допускаю\newline
    %     \hbox to 0.4\textwidth{<<\hbox to 12mm{\hrulefill}>> \hrulefill~2010 г.}
    %     &
    %     К защите рекомендую\newline
    %     \hbox to 0.45\textwidth{<<\hbox to 12mm{\hrulefill}>> \hrulefill~2010 г.}
    % \end{tabular}

    \vspace{\fill}
    Москва, 2021
\end{center}
\end{titlepage}
\newpage
\tableofcontents
\newpage

\section{Введение}
TL;DR мотивация: все постоянно настраивают модели на accuracy, а на корректность предсказанных вероятностей забивают. Ответить: Зачем нужны корректные вероятности? Примеры (прогноз погоды, использование в других моделях, машинный перевод, healthcare, self-driving cars). Если будет много, то вынести в отдельный раздел (like <<Применение>>).
\section{Оценка откалиброванности}
% переназвать
Reliability (calibration) diagrams (plots/curves). Метрики. Их беды.
\section{Методы}
Калибровка является техникой пост-обработки. Если модель заведомо учится предсказывать корректные вероятности (или звезды сошлись так, что она откалибрована -- пример в статье о температурном шкалировании), то данные методы могут быть лишними. Какой использовать? Зависит от задачи и классификатора. А также нужно смотреть на диаграмму.
<<Функция трансформации>>. Параметрические, непараметрические методы. 
\subsection{Бинарная задача}
\subsection{Многоклассовая}
\section{Вычислительные эксперименты}
\section{Заключение}
\section{Список литературы}
\section{Прочие идеи}

\end{document}